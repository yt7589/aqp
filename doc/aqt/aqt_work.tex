\documentclass{article}
\usepackage{xeCJK}
% if you need to pass options to natbib, use, e.g.:
%\PassOptionsToPackage{numbers, compress}{natbib}
% before loading nips_2017
%
% to avoid loading the natbib package, add option nonatbib:
% \usepackage[nonatbib]{nips_2017}

\usepackage[final]{nips_2017}

% to compile a camera-ready version, add the [final] option, e.g.:
% \usepackage[final]{nips_2017}

\usepackage[utf8]{inputenc} % allow utf-8 input
\usepackage[T1]{fontenc}    % use 8-bit T1 fonts
\usepackage{hyperref}       % hyperlinks
\usepackage{url}            % simple URL typesetting
\usepackage{booktabs}       % professional-quality tables
\usepackage{amsfonts}       % blackboard math symbols
\usepackage{nicefrac}       % compact symbols for 1/2, etc.
\usepackage{microtype}      % microtypography
\usepackage{bm}             % bold in math
\usepackage{graphicx}       % images
\usepackage{algorithm}      % algorithm
\usepackage[noend]{algpseudocode} % algorithm
\usepackage{caption}        % captionof
\usepackage{array}          % thick column hline
\usepackage{booktabs}       % table style
\usepackage{pbox}           % table line break
\usepackage{subcaption}     % multiple figures
\usepackage{listings}
\usepackage{xcolor}
\usepackage{tikz}
\usepackage{amsmath}
\definecolor{mygreen}{rgb}{0,0.6,0}
\definecolor{mygray}{rgb}{0.5,0.5,0.5}
\definecolor{mymauve}{rgb}{0.58,0,0.82}
\definecolor{codeBkg}{rgb}{0.85,0.85,0.85}

\lstset{ 
	backgroundcolor=\color{codeBkg},   % choose the background color; you must add \usepackage{color} or \usepackage{xcolor}; should come as last argument
	basicstyle=\footnotesize,        % the size of the fonts that are used for the code
	breakatwhitespace=false,         % sets if automatic breaks should only happen at whitespace
	breaklines=true,                 % sets automatic line breaking
	captionpos=b,                    % sets the caption-position to bottom
	commentstyle=\color{mygreen},    % comment style
	deletekeywords={...},            % if you want to delete keywords from the given language
	escapeinside={\%*}{*)},          % if you want to add LaTeX within your code
	extendedchars=true,              % lets you use non-ASCII characters; for 8-bits encodings only, does not work with UTF-8
	frame=no,	                   % adds a frame around the code
	keepspaces=true,                 % keeps spaces in text, useful for keeping indentation of code (possibly needs columns=flexible)
	keywordstyle=\color{blue},       % keyword style
	language=Octave,                 % the language of the code
	morekeywords={*,...},            % if you want to add more keywords to the set
	numbers=left,                    % where to put the line-numbers; possible values are (none, left, right)
	numbersep=5pt,                   % how far the line-numbers are from the code
	numberstyle=\tiny\color{mygray}, % the style that is used for the line-numbers
	rulecolor=\color{black},         % if not set, the frame-color may be changed on line-breaks within not-black text (e.g. comments (green here))
	showspaces=false,                % show spaces everywhere adding particular underscores; it overrides 'showstringspaces'
	showstringspaces=false,          % underline spaces within strings only
	showtabs=false,                  % show tabs within strings adding particular underscores
	stepnumber=1,                    % the step between two line-numbers. If it's 1, each line will be numbered
	stringstyle=\color{mymauve},     % string literal style
	tabsize=4,	                   % sets default tabsize to 2 spaces
	title=\lstname                   % show the filename of files included with \lstinputlisting; also try caption instead of title
}

\title{高级量化交易技术}

\hypersetup{
    colorlinks = true,
}
\makeatletter
\def\BState{\State\hskip-\ALG@thistlm}
\makeatother
\floatname{algorithm}{Procedure}
\renewcommand{\algorithmicrequire}{\textbf{Input:}}
\renewcommand{\algorithmicensure}{\textbf{Output:}}

\newcolumntype{?}{!{\vrule width 3pt}}

\author{
  闫涛 \\
  %% examples of more authors
  %% \And
  %% Nicholas Frosst \\
  %% Affiliation \\
  %% Address \\
  %% \texttt{email} \\
  %% \AND
  %% Geoffrey E. Hinton \\
  科技有限公司\\
  北京 \\
  \texttt{\{yt7589\}@qq.com} \\
  %% Affiliation \\
  %% Address \\
  %% \texttt{email} \\
  %% \And
  %% Coauthor \\
  %% Affiliation \\
  %% Address \\
  %% \texttt{email} \\
  %% \And
  %% Coauthor \\
  %% Affiliation \\
  %% Address \\
  %% \texttt{email} \\
}
\date{March 2018}

% \usepackage{natbib}
% \usepackage{graphicx}

\begin{document}






\maketitle
\begin{center}
\Large \textbf{第1章} \quad \textbf{时间序列基本特性}
\end{center}
\begin{abstract}
在本章中我们将讨论时间序列的基本特性,包括自相关性和平稳性。
\end{abstract}
\section{时间序列基本特性}
时间序列的自相关性是指时间序列过去与未来存在某种关系,是我们时间序列预测的基础。主要用自协方差函数(Autocovariance Function, AF)、自相关系数函数(Autocorrelation Coefficent Function, ACF)和偏自相关系数函数(Partial Autocorrelation Coefficient Function, PACF)来描述。


\subsection{启动过程}
首先是FmeEngine的构造函数,如下所示:
\lstset{language=PYTHON, caption={FmeEngine的构造函数}, label={c000086}}
\begin{lstlisting}
class FmeEngine(object):
    def __init__(self):
        self.name = 'FmeEngine'
        self.env = None
        self.agent = FmeXgbAgent()
        self.test_size = 1000
\end{lstlisting}
这里面重要的代码是初始化一个FmeXgbAgent类的实例,该实例利用XGBoost算法来选择策略。下面来看FmeXgbAgent类的构造函数:
\lstset{language=PYTHON, caption={FmeXgbAgent的构造函数}, label={c000087}}
\begin{lstlisting}
class FmeXgbAgent(object):
    def __init__(self):
        self.name = 'FmeXgbAgent'
        self.model_file = './work/btc_drl.xgb'
        self.max_min_file = './work/btc_max_min.csv'
        self.bst = None
        self.fme_dataset = FmeDataset()
        self.X, self.y = self.fme_dataset.load_bitcoin_dataset()
        self.model = self.train_baby_agent()
        self.df = None
        self.fme_env = None
        self.max_min_file = './work/btc_max_min.csv'
        self.dataset_size = 10
        self.cached_quotation = np.loadtxt(self.max_min_file, delimiter=',')
\end{lstlisting}
这段代码中,最重要的是调用train\_baby\_agent方法,其用前1000个时间点,训练一个初始化的XGBoost模型,该模型可以选择适合的
动作。这部分代码在之前已经讲述过,这里就不再复述了。\newline
系统程序入口在FmeEngine.startup方法中,如下所示:
\lstset{language=PYTHON, caption={程序入口点}, label={c000082}}
\begin{lstlisting}
def startup(self):
    self.env = self.build_raw_env()
    self.agent.df = self.fme_env.df
    self.agent.fme_env = self.fme_env
    obs = self.env.reset()
    for i in range(self.slice_point):
        action = self.agent.choose_action(i+self.fme_env.lookback_window_size, obs)
        obs, rewards, done, info = self.env.step([action])
        if done:
            break
        self.env.render(mode="human", title="BTC")
        # 重新训练模型
        self.agent.train_drl_agent(info[0]['weight'])
    print('回测结束 ^_^')
\end{lstlisting}
\begin{itemize}
\item 第2行:创建深度强化学习环境,以比特币分钟级数据为环境,划分训练样本集和测试样本集;
\item 第3、4行:在Agent中保存数据集内容和环境,主要是便于进行训练;
\item 第5行:重置环境;
\item 第7$\sim$13行:每个时间点之前的几个时间点(5个)的数据组成一个样本,也是环境的一个状态,Agent将根据
这个状态决定要采取的行动:买入、持有、卖出,在操作之后,根据新的净资产与原来净资产的大小,决定奖励信号,奖励信号
为新净资产与老净资产的比值。以上为深度强化学习的一步,循环执行此过程,直到运行完所有时间点。在每个时间点,第11行
以图形方式绘制交易情况和净资产变化情况,我们将奖励信号作为对应样本的权重,重新训练我们策略网络,这里我们用的是
XGBoost。
\end{itemize}
深度强化学习中环境是一个非常重要的因素,我们来具体看一下环境的创建过程:
\lstset{language=PYTHON, caption={深度强化学习环境创建}, label={c000083}}
\begin{lstlisting}
def build_raw_env(self):
    ''' 创建原始比特币行情文件生成的env,主要用于深度强化学习试验 '''
    self.df = pd.read_csv('./data/bitstamp.csv')
    self.df = self.df.drop(range(FmeDataset.DATASET_SIZE))
    self.df = self.df.dropna().reset_index()
    self.df = self.df.sort_values('Timestamp')
    self.agent.df = self.df
    self.slice_point = int(len(self.df) - self.test_size)
    self.train_df = self.df[:self.slice_point]
    self.test_df = self.df[self.slice_point:]
    self.fme_env = FmeEnv(self.train_df, serial=True)
    self.agent.fme_env = self.fme_env
    return DummyVecEnv(
        [lambda: self.fme_env])
\end{lstlisting}
\begin{itemize}
\item 第3行:从CSV文件中读出DataFrame格式的比特币分钟级数据;
\item 第4行:忽略前1000个时间点,这些时间点将用于训练一个初始的Policy Gradient模型,加快学习进程;
\item 第5、6行:去掉为空的行并重建索引,同时按时间进行排序;
\item 第8行:slice\_point是训练样本集和测试样本集的分隔点;
\item 第11行:是本段程序的重点,其初始化了一个深度强化学习环境;
\end{itemize}
下面我们来看深度强化学习环境的定义,如下所示:
\lstset{language=PYTHON, caption={深度强化学习环境类构造函数}, label={c000084}}
\begin{lstlisting}
MAX_TRADING_SESSION = 100000

class FmeEnv(gym.Env):
    def __init__(self, df, lookback_window_size=50,
        commission=0.00075, initial_balance=10000,
        serial=False
    ):
        self.name = 'FmeEnv'
        print('Finacial Market Env is starting up...')
        random.seed(100)
        self.buy_rate = 1.0 # 20%机会购买
        self.sell_rate = 1.0 # 15%机会卖
        self.df = df.dropna().reset_index()
        print(self.df.head(10))
        self.lookback_window_size = lookback_window_size
        self.initial_balance = initial_balance
        self.commission = commission
        self.serial = serial  # Actions of the format Buy 1/10, Sell 3/10, Hold, etc.
        # Observes the OHCLV values, net worth, and trade history
        self.scaler = preprocessing.MinMaxScaler()
        self.viewer = None
        self.action_space = spaces.MultiDiscrete([3, 10])
        self.observation_space = spaces.Box(low=0, high=1, shape=(10, 
                    lookback_window_size + 1), dtype=np.float16)
\end{lstlisting}
由上面的代码可以看出,该类继承自gym.Env,除了构造函数外,还有reset和step是需要重载的方法,我们将在后面的流程中
进行讲解。\newline
如表\ref{c000082}系统会首先调用FmeEngine.reset方法,初始化环境,并返回环境的初始状态。下面我们来看FmeEnv类的reset
方法:
\lstset{language=PYTHON, caption={深度强化学习环境类重置方法}, label={c000085}}
\begin{lstlisting}
def reset(self):
    self.balance = self.initial_balance
    self.net_worth = self.initial_balance
    self.btc_held = 0
    self._reset_session()
    self.account_history = np.repeat([
        [self.balance],
        [0],
        [0],
        [0],
        [0]
    ], self.lookback_window_size + 1, axis=1)
    self.trades = []
    return self._next_observation()

def _reset_session(self):
    self.current_step = 0
    if self.serial:
        self.steps_left = len(self.df) - self.lookback_window_size - 1
        self.frame_start = self.lookback_window_size
    else:
        self.steps_left = np.random.randint(1, MAX_TRADING_SESSION)
        self.frame_start = np.random.randint(
            self.lookback_window_size, len(self.df) - self.steps_left)
    self.active_df = self.df[self.frame_start - self.lookback_window_size:
                                self.frame_start + self.steps_left]

def _next_observation(self):
    end = self.current_step + self.lookback_window_size + 1
    scaled_df = self.active_df.values[:end].astype(np.float64)
    scaled_df = self.scaler.fit_transform(scaled_df)
    scaled_df = pd.DataFrame(scaled_df, columns=self.df.columns)
    obs = np.array([
        scaled_df['Open'].values[self.current_step:end],
        scaled_df['High'].values[self.current_step:end],
        scaled_df['Low'].values[self.current_step:end],
        scaled_df['Close'].values[self.current_step:end],
        scaled_df['Volume_(BTC)'].values[self.current_step:end],
    ])
    scaled_history = self.scaler.fit_transform(self.account_history.astype(np.float64))
    obs = np.append(
        obs, scaled_history[:, -(self.lookback_window_size + 1):], axis=0)
    return obs
\end{lstlisting}
\begin{itemize}
\item 第2$\sim$4行:重置资金余额、资产净值和比特币持有量;
\item 第5行:重置session,这里的session与监督学习中的epoch实际上是同一概念,就是遍历所有时间点;
\item   \begin{itemize}
        \item 第17行:current\_step是当前时间点的指针,指向当前时间点,初始值指向第1个时间点;
        \item 第18$\sim$20行:self.serial代表是否从第一个时间点开始,如果是否的话,则从一个随机的时间点
        开始;这里是从第1个时间点开始的情况,lookback\_window\_size表示从当前时间点开始,向前取几个时间
        点形成一个样本数据用于进行操作选择,因此开始时间点应该从第lookback\_window\_size开始,结束时间点
        应该到最后一个时间点前的lookback\_window\_size个时间点结束;
        \item 第21$\sim$24行:处理从随机的时间点开始的情况;
        \item 第25、26行:定义active\_df是活跃的时间点记录;
        \end{itemize}
\item 第6$\sim$12行:设置账户的操作历史,在每个时间点,账户历史信息包括:余额、买入量、买入金额、卖出量、卖出金额;
\item 第13行:清空交易历史;
\item 第14行:向Agent返回当前状态;
\item 
    \begin{itemize}
    \item 第29行:求出当前状态的结束时间点end;
    \item 第30$\sim$32行:对数据进行归一化,设训练样本集最大值为$v_{max}$,最小值为$v_{min}$,公式为$\hat{v}=\frac{v - v_{min}}{v_{max}-v_{min}}$,归一化为$[0,1]$之间的数,这种方法的缺点是对异常点数据敏感,但是比特币交易数据很少会出现异常数据;
    \item 第33$\sim$39行:将开盘价、最高价、最低价、收盘价、成交量所有时间点的数据分别作为一行(与数据集要求每一行代表一个样本相反);
    \item ??????? 作用以后补全;
    \end{itemize}
\end{itemize}
在一个Session中,对于每个时间点,我们首先通过FmeXgbAgent来选择合适的操作,代码如下所示:
\lstset{language=PYTHON, caption={FmeXgbAgent选择操作}, label={c000088}}
\begin{lstlisting}
def choose_action(self, idx, obs):
    '''  '''
    commission = self.fme_env.commission
    frame_size = self.fme_dataset.frame_size
    recs = self.df.iloc[idx-frame_size+1:idx+1]
    datas = np.array(recs)
    ds = datas[:, 3:8]
    print('ds.shape:{0}; frame_size={1}; idx={2}'.format(ds.shape, frame_size, idx))
    ds = np.reshape(ds, (frame_size*5, ))
    date_quotation = ds[20:25]
    if self.fme_env.btc_held <= 0.00000001:
        x = np.append(ds, [0.0])
    else:
        x = np.append(ds, [1.0])
    self.add_quotation_tick(self.cached_quotation, [x[20], x[21], x[22], x[23], x[24]])
    ds_max = np.amax(self.cached_quotation, axis=0)
    ds_min = np.amin(self.cached_quotation, axis=0)
    self.normalize_ds(x, ds_max, ds_min)
    print('x:{0:04f}, {1:04f}, {2:04f}, {3:04f}, {4:04f}, {5:04f}, '
            '{6:04f}, {7:04f}, {8:04f}, {9:04f}, {10:04f}, {11:04f},'
            '{12:04f}, {13:04f}, {14:04f}, {15:04f}, {16:04f}, {17:04f}, {18:04f},'
            '{19:04f}, {20:04f}, {21:04f}, {22:04f}, {23:04f}, {24:04f}, {25:04f},'.format(
                x[0], x[1], x[2], x[3], x[4], 
                x[5], x[6], x[7], x[8], x[9],
                x[10], x[11], x[12], x[13], x[14],
                x[15], x[16], x[17], x[18], x[19],
                x[20], x[21], x[22], x[23], x[24], x[25]
            ))
    xg = xgb.DMatrix([x], label=x)
    pred = self.model.predict(xg)
    action_type = np.argmax(pred)
    print('pred:{0}; [{1:02f}, {2:02f}, {3:02f}]=>{4}'.format(
        pred.shape, pred[0][0], pred[0][1], pred[0][2],
        np.argmax(pred[0]))
    )
    if 0 == action_type:
        action = np.array([0, 10])
    elif 1 == action_type:
        action = np.array([1, 10])
    else:
        action = np.array([2, 10])
    self.x = x
    self.action = action_type
    return action
\end{lstlisting}
\begin{itemize}
\item 第3行:指定手续费费率;
\item 第4行:frame\_size表示从当前时间点向前看几个时间点,由这几个时间点的行情数据来选择最有利的操作;
\item 第5行:idx参数代表当前时间点,一个样本的数据由当前时间点之前frame\_size个时间点的数据再加上当前时间点的数据,组成一个样本;
\item 
\end{itemize}



产品经理常犯的错误:
1 自我感觉良好
2 知其然,不知其所以然
3 老板的话是圣旨
4 需求变更频繁
5 不善于沟通
6 不重视需求文档和原型
7 为了做产品而做产品,没有反思和复盘
8 项目管理混乱
9 不做计划和总结




\section{汇总}
\lstset{language=BASH, caption={编号}, label={c900038}}
\begin{lstlisting}
t000004
f000022
c000089
e000122
\end{lstlisting}

参考文献:\cite{r000001}


\subsubsection{Transformer策略}
生成训练样本集,采用MLP模型进行训练:\newline
取出lookback\_window\_size+1条数据,运行choose\_action算法,求出其应该是[0,0,0],分别对应于买入、卖出、持有,
然后采用TensorFlow 2.0 MLP算法作为交易策略并进行训练,将choose\_action替换为MLP的predict方法,然后运行
测试样本集,得到最终的回测结果。必须包括当前的余额值,所以其有两个记录,分别对应满仓和空仓时的操作。\newline
将策略算法换为XGBoost。\newline
将策略算法换为Transformer。\newline



\section{XLNet模型}









\lstset{language=BASH, caption={参考链接}, label={c900037}}
\begin{lstlisting}
\text{数学公式1}https://meta.wikimedia.org/wiki/Help:Displaying_a_formula
https://arxiv.org/pdf/1805.09692.pdf
https://www.biorxiv.org/content/biorxiv/early/2018/07/03/360537.full.pdf
https://github.com/tensorflow/tensor2tensor/blob/master/tensor2tensor/models/evolved_transformer.py#L66

\end{lstlisting}





\newpage

\bibliographystyle{plainnat}
\bibliography{nips}

\appendix


\end{document}



